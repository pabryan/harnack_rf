\documentclass{amsart}

\input{StandardPaper2.tex}

\begin{document}

\title[Harnack For Ricci Flow]
 {The Ricci Flow Harnack Inequality Via Reparametrisation}

\curraddr{}
\email{}
\date{\today}

\dedicatory{}
\subjclass[2010]{}
\keywords{}

\begin{abstract}
\end{abstract}

\maketitle

\section{Introduction}
\label{sec:intro}

The Ricci flow is the highly studied evolution equation,
\[
\partial_t g_t = - 2\ric (g_t)
\]
for a smooth, one-parameter family of metrics \(g_t\) on a smooth manifold \(M^n\) of dimension \(n\).

\section{Notation and Conventions}
\label{sec:conventions}

\subsection{Curvature Conventions}
\label{subsec:conventions_curvature}

We take the convention for the curvature tensor that,
\[
\Rm(X, Y) Z = \nabla_X \nabla_Y - \nabla_ Y \nabla_X - \nabla_{[X, Y]} Z = \nabla^2_{X, Y} Z - \nabla^2_{Y, X} Z
\]
where
\[
\nabla^2_{X, Y} Z = \nabla^2 Z (X, Y) = \nabla_X \nabla_Y Z - \nabla_{\nabla_X Y} Z
\]
is the second covariant derivative of \(Z\). The \((0, 4)\) metric contraction of \(\Rm\) is defined by
\[
\Rm(X, Y, Z, W) = g(\Rm(X, Y)Z, W).
\]

The Ricci tensor is the trace in the first position:
\[
\ric(X, Y) = \Tr (Z \mapsto \Rm(Z, X)Y)
\]
Since the metric is non-degenerate, we can define the Ricci operator \(\ricop\), by
\[
\ric(X, Y) = g(\ricop(X), Y) = g(X, \ricop(Y))
\]
the second equality because the symmetries of \(\Rm\) imply that \(\ric\) is symmetric. The scalar curvature is then the trace of \(\ricop\):
\[
\scalarcurv = \Tr \ricop.
\]

With respect to a frame \(\{e_i\}\) with dual frame \(\{\theta^i\}\) we write the curvature tensor as,
\[
\Rm = {\Rm_{ijk}}^l \theta^i \otimes \theta^j \otimes \theta^k \otimes e_l
\]
so that
\[
{\Rm_{ijk}}^l = (\Rm(e_i, e_j) e_k) (\theta^l) = \theta^l (\Rm(e_i, e_j) e_k)
\]
where in the first equality we identify the vector \(\Rm(e_i, e_j)e_k\) as an element of the double dual via the formula \(X(\alpha) = \alpha(X)\) for \(X\) a vector and \(\alpha\) a one-form.

The metric contraction is then on the last index,
\[
\Rm_{ijkl} = \Rm(e_i, e_j, e_k, e_l) = g(\Rm(e_i, e_j) e_k, e_l) = {\Rm_{ijk}}^mg_{ml}
\]
with reverse contraction,
\[
{\Rm_{ijk}}^l = \Rm_{ijkm} g^{ml}.
\]
The Ricci tensor is then
\[
\ric_{ij} = \ric(e_i, e_j) = {\Rm_{lij}}^l = \Rm_{lijm} g^{ml}.
\]
Notice we trace on the first and fourth slot of \(\Rm\).

\subsection{Basic Evolution Equations}
\label{subsec:evolution}

We require remarkably little evolution equations to obtain the Harnack. All we require is that under the Ricci flow we have,
\begin{align}
\partial_t \R &= \laplace \R + 2 \abs{\ric}^2 \label{eq:R_ev} \\
[\partial_t, \laplace] f &= 2 g(\ric, \nabla^2 f) \label{eq:R_commutator}
\end{align}
for any smooth function \(f\).

\section{The Harnack Inequality}
\label{sec:harnack}

\subsection{Reparametrisation}
\label{subsec:harnack_reparam}

\subsection{Li-Yau Harnack Inequality}
\label{subsec:harnack_liyau}

\end{document}
