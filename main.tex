\documentclass{amsart}

%\usepackage{etoolbox}
%\makeatletter
%\let\ams@starttoc\@starttoc
%\makeatother
%\makeatletter
%\let\@starttoc\ams@starttoc
%\patchcmd{\@starttoc}{\makeatletter}{\makeatletter\parskip\z@}{}{}
%\makeatother

%\usepackage[parfill]{parskip}

\usepackage[colorlinks=true,linkcolor=blue,citecolor=blue,urlcolor=blue]{hyperref}
\usepackage{bookmark}
\usepackage{amsthm,thmtools,amssymb,amsmath,amscd}

\usepackage[bibstyle=alphabetic,citestyle=alphabetic,backend=bibtex]{biblatex}
\bibliography{Bibliography}

\usepackage{fancyhdr}
\usepackage{esint}

\usepackage{enumerate}

\usepackage{pictexwd,dcpic}

\usepackage{graphicx}

\swapnumbers
\declaretheorem[name=Theorem,numberwithin=section]{thm}
\declaretheorem[name=Remark,style=remark,sibling=thm]{rem}
\declaretheorem[name=Lemma,sibling=thm]{lemma}
\declaretheorem[name=Proposition,sibling=thm]{prop}
\declaretheorem[name=Definition,style=definition,sibling=thm]{defn}
\declaretheorem[name=Corollary,sibling=thm]{cor}
\declaretheorem[name=Assumption,style=remark,sibling=thm]{ass}
\declaretheorem[name=Example,style=remark,sibling=thm]{example}


\numberwithin{equation}{section}

\usepackage{cleveref}
\crefname{lemma}{Lemma}{Lemmata}
\crefname{prop}{Proposition}{Propositions}
\crefname{thm}{Theorem}{Theorems}
\crefname{cor}{Corollary}{Corollaries}
\crefname{defn}{Definition}{Definitions}
\crefname{example}{Example}{Examples}
\crefname{rem}{Remark}{Remarks}
\crefname{ass}{Assumption}{Assumptions}
\crefname{not}{Notation}{Notation}

%Symbols
\renewcommand{\~}{\tilde}
\renewcommand{\-}{\bar}
\newcommand{\bs}{\backslash}
\newcommand{\cn}{\colon}
\newcommand{\sub}{\subset}

\newcommand{\NN}{\mathbb{N}}
\newcommand{\RR}{\mathbb{R}}
\newcommand{\ZZ}{\mathbb{Z}}
\renewcommand{\SS}{\mathbb{S}}
\newcommand{\HH}{\mathbb{H}}
\newcommand{\CC}{\mathbb{C}}
\newcommand{\KK}{\mathbb{K}}
\newcommand{\Di}{\mathbb{D}}
\newcommand{\BB}{\mathbb{B}}
\newcommand{\8}{\infty}

%Greek letters
\renewcommand{\a}{\alpha}
\renewcommand{\b}{\beta}
\newcommand{\g}{\gamma}
\renewcommand{\d}{\delta}
\newcommand{\e}{\epsilon}
\renewcommand{\k}{\kappa}
\renewcommand{\l}{\lambda}
\renewcommand{\o}{\omega}
\renewcommand{\t}{\theta}
\newcommand{\s}{\sigma}
\newcommand{\p}{\varphi}
\newcommand{\z}{\zeta}
\newcommand{\vt}{\vartheta}
\renewcommand{\O}{\Omega}
\newcommand{\D}{\Delta}
\newcommand{\G}{\Gamma}
\newcommand{\T}{\Theta}
\renewcommand{\L}{\Lambda}

%Mathcal Letters
\newcommand{\cL}{\mathcal{L}}
\newcommand{\cT}{\mathcal{T}}
\newcommand{\cA}{\mathcal{A}}
\newcommand{\cW}{\mathcal{W}}

%Mathematical operators
\newcommand{\INT}{\int_{\O}}
\newcommand{\DINT}{\int_{\d\O}}
\newcommand{\Int}{\int_{-\infty}^{\infty}}
\newcommand{\del}{\partial}
\DeclareMathOperator{\lie}{\mathcal{L}}

\newcommand{\inpr}[2]{\left\langle #1,#2 \right\rangle}
\newcommand{\fr}[2]{\frac{#1}{#2}}
\newcommand{\x}{\times}
\DeclareMathOperator{\Tr}{Tr}

\DeclareMathOperator{\dive}{div}
\DeclareMathOperator{\id}{id}
\DeclareMathOperator{\pr}{pr}
\DeclareMathOperator{\Diff}{Diff}
\DeclareMathOperator{\supp}{supp}
\DeclareMathOperator{\graph}{graph}
\DeclareMathOperator{\osc}{osc}
\DeclareMathOperator{\const}{const}
\DeclareMathOperator{\dist}{dist}
\DeclareMathOperator{\loc}{loc}
\DeclareMathOperator{\grad}{grad}
\DeclareMathOperator{\ric}{Ric}
\DeclareMathOperator{\Rm}{Rm}
\DeclareMathOperator{\R}{R}
\DeclareMathOperator{\scalarcurv}{S}
\DeclareMathOperator{\weingarten}{\mathfrak{W}}
\DeclareMathOperator{\inj}{inj}
\DeclareMathOperator{\einstein}{E}
\DeclareMathOperator{\scalareinstein}{\mathcal{E}}

\newcommand{\op}[1]{\mathfrak{#1}}
\DeclareMathOperator{\ricop}{\op{Ric}}
\DeclareMathOperator{\einsteinop}{\op{E}}

\newcommand{\dual}[1]{{#1}^{\ast}}
\newcommand{\abs}[1]{\left|{#1}\right|}
\DeclareMathOperator{\sgn}{sgn}

%Environments
\newcommand{\Theo}[3]{\begin{#1}\label{#2} #3 \end{#1}}
\newcommand{\pf}[1]{\begin{proof} #1 \end{proof}}
\newcommand{\eq}[1]{\begin{equation}\begin{alignedat}{2} #1 \end{alignedat}\end{equation}}
\newcommand{\IntEq}[4]{#1&#2#3	 &\quad &\text{in}~#4,}
\newcommand{\BEq}[4]{#1&#2#3	 &\quad &\text{on}~#4}
\newcommand{\br}[1]{\left(#1\right)}



%Logical symbols
\newcommand{\Ra}{\Rightarrow}
\newcommand{\ra}{\rightarrow}
\newcommand{\hra}{\hookrightarrow}
\newcommand{\mt}{\mapsto}

% Aleksandrov Reflection Macros
\DeclareMathOperator{\reflectionvector}{V}
\DeclareMathOperator{\reflectionangle}{\delta}
\newcommand{\reflectionplane}[1][\reflectionvector]{\ensuremath{P_{#1}}}
\newcommand{\reflectionmap}[1][\reflectionvector]{\ensuremath{R_{#1}}}
\newcommand{\reflectionset}[2][\reflectionvector]{\ensuremath{{#2}_{#1}}}
\newcommand{\reflectionhalfspace}[1][\reflectionvector]{\ensuremath{\reflectionset[{#1}]{H}}}
\DeclareMathOperator{\vertvec}{e}
\DeclareMathOperator{\origin}{O}
\DeclareMathOperator{\radialprojection}{\pi}
\DeclareMathOperator{\height}{h}
\DeclareMathOperator{\equator}{E}
\newcommand{\ip}[2]{\ensuremath{\langle{#1},{#2}\rangle}}
\DeclareMathOperator{\intersect}{\cap}
\DeclareMathOperator{\union}{\cup}
\DeclareMathOperator{\nor}{\nu}
\DeclareMathOperator{\basepoint}{p_0}
\DeclareMathOperator{\radialdistance}{r}

%Fonts
\newcommand{\mc}{\mathcal}
\renewcommand{\it}{\textit}
\newcommand{\mrm}{\mathrm}

%Spacing
\newcommand{\hp}{\hphantom}


%\parindent 0 pt

\protected\def\ignorethis#1\endignorethis{}
\let\endignorethis\relax
\def\TOCstop{\addtocontents{toc}{\ignorethis}}
\def\TOCstart{\addtocontents{toc}{\endignorethis}}


\begin{document}

\title[Harnack For Ricci Flow]
 {The Ricci Flow Harnack Inequality Via Reparametrisation}

\curraddr{}
\email{}
\date{\today}

\dedicatory{}
\subjclass[2010]{}
\keywords{}

\begin{abstract}
\end{abstract}

\maketitle

\section{Introduction}
\label{sec:intro}

The Ricci flow is the highly studied evolution equation,
\[
\partial_t g_t = - 2\ric (g_t)
\]
for a smooth, one-parameter family of metrics \(g_t\) on a smooth manifold \(M^n\) of dimension \(n\).

We prove the Trace-Harnack inequality for the Ricci flow.

\begin{thm}[Trace Harnack Inequality For Ricci Flow\cite{Hamilton:/1993}]
\label{thm:harnack}
Let \((M, g)\) be a solution of the Ricci flow with \(\ric \geq K_0 g\) for some \(K_0 > 0\). Then,
\[
\partial_t \R - \abs{\nabla \R}_{\ric}^2 + \frac{\R}{t} \geq 0.
\]
\end{thm}

\section{Notation and Conventions}
\label{sec:conventions}

\subsection{Curvature Conventions}
\label{subsec:conventions_curvature}

We take the convention for the curvature tensor that,
\[
\Rm(X, Y) Z = \nabla_X \nabla_Y - \nabla_ Y \nabla_X - \nabla_{[X, Y]} Z = \nabla^2_{X, Y} Z - \nabla^2_{Y, X} Z
\]
where
\[
\nabla^2_{X, Y} Z = \nabla^2 Z (X, Y) = \nabla_X \nabla_Y Z - \nabla_{\nabla_X Y} Z
\]
is the second covariant derivative of \(Z\). The \((0, 4)\) metric contraction of \(\Rm\) is defined by
\[
\Rm(X, Y, Z, W) = g(\Rm(X, Y)Z, W).
\]

The Ricci tensor is the trace in the first position:
\[
\ric(X, Y) = \Tr (Z \mapsto \Rm(Z, X)Y)
\]
Since the metric is non-degenerate, we can define the Ricci operator \(\ricop\), by
\[
\ric(X, Y) = g(\ricop(X), Y) = g(X, \ricop(Y))
\]
the second equality because the symmetries of \(\Rm\) imply that \(\ric\) is symmetric. The scalar curvature is then the trace of \(\ricop\):
\[
\scalarcurv = \Tr \ricop.
\]

With respect to a frame \(\{e_i\}\) with dual frame \(\{\theta^i\}\) we write the curvature tensor as,
\[
\Rm = {\Rm_{ijk}}^l \theta^i \otimes \theta^j \otimes \theta^k \otimes e_l
\]
so that
\[
{\Rm_{ijk}}^l = (\Rm(e_i, e_j) e_k) (\theta^l) = \theta^l (\Rm(e_i, e_j) e_k)
\]
where in the first equality we identify the vector \(\Rm(e_i, e_j)e_k\) as an element of the double dual via the formula \(X(\alpha) = \alpha(X)\) for \(X\) a vector and \(\alpha\) a one-form.

The metric contraction is then on the last index,
\[
\Rm_{ijkl} = \Rm(e_i, e_j, e_k, e_l) = g(\Rm(e_i, e_j) e_k, e_l) = {\Rm_{ijk}}^mg_{ml}
\]
with reverse contraction,
\[
{\Rm_{ijk}}^l = \Rm_{ijkm} g^{ml}.
\]
The Ricci tensor is then
\[
\ric_{ij} = \ric(e_i, e_j) = {\Rm_{lij}}^l = \Rm_{lijm} g^{ml}.
\]
Notice we trace on the first and fourth slot of \(\Rm\).

\subsection{Basic Evolution Equations}
\label{subsec:evolution}

We require remarkably little evolution equations to obtain the Harnack. All we require is that under the Ricci flow we have,
\begin{align}
\partial_t \R &= \laplace \R + 2 \abs{\ric}^2 \label{eq:R_ev} \\
[\partial_t, \laplace] f &= 2 g(\ric, \nabla^2 f) \label{eq:R_commutator}
\end{align}
for any smooth function \(f\).

\section{The Harnack Inequality}
\label{sec:harnack}

Our approach here is motivated by the differential Harnack inequality of \cite{LiYau:/1986} (see also \cite{Hamilton:/1986,HamiltonCao:/2009}). Let us define \(u = \ln \R\). Then the Harnack inequality is equivalent to
\begin{equation}
\label{eq:harnack_equiv}
\partial_t u - \frac{\R}{2} \abs{\nabla u}_{\ric}^2 \geq - \frac{1}{t}.
\end{equation}

Recalling the evolution of \(\R\) \eqref{eq:R_ev}, let us define
\begin{equation}
\label{eq:Q}
Q = \frac{1}{\R} \left(\laplace \R + 2\abs{\ric}^2\right) - \frac{\R}{2} \abs{\nabla u}_{\ric}^2
\end{equation}
so that the Harnack inequality becomes,
\[
Q \geq - \frac{1}{t}.
\]
The strategy is to compute \((\partial_t - \laplace) Q\) and apply the maximum principle. The second term in the definition of \(Q\) \eqref{eq:Q} is required because of the evolution of the metric and is not present in \cite{LiYau:/1986}. Computing \((\partial_t - \Delta)\) of this term is rather tedious, so next we introduce a reparametrisation \(\bar{g} = \varphi_t^{\ast} g\) in which this term dissapears and the Harnack inequality is transformed to
\[
\partial_t \bar{u} \geq - \frac{1}{t}.
\]

\subsection{Reparametrisation}
\label{subsec:harnack_reparam}

The key step to make the computation significantly more tractable is to reparametrise the flow. From the assumption \(\ric \geq K_0 g\) with \(K_0 > 0\) we see that \(\ric\) is a metric. Hence we may define,
\[
\xi_t = \frac{1}{2} \grad_{\ric_t} \R_t
\]
by the requirement that,
\[
\ric_t (\xi_t, X) = \frac{1}{2} d\R_t \cdot X
\]
for any tangent vector \(X\).

Now let \(\varphi_t\) be the flow of \(X_t\) starting at \(t = 0\), namely the smooth isotopy such that \(\varphi_0 = \Id_M\) and \(\partial_t \varphi_t = \xi_t\). Define
\[
\bar{g}_t = \varphi_t^{\ast} g_t.
\]

Given a tensor \(T\) derived from \(g\) we write \(\bar{T} = \phi_t^{\ast} T\) for the corresponding tensor derived from \(\bar{g}\). In particular \(\bar{\R} = \phi_t^{\ast} \R = \R(\bar{g})\), \(\bar{\ric} = \phi_t^{\ast} \ric = \ric(\bar{g})\), and \(\bar{u} = \phi_t^{\ast} u = \ln \bar{\R} = \ln \R(\bar{g})\).

When there are time derivatives involved, such as in the definition of \(\partial_t u\) we must also account for the time variation. First, the metric \(\bar{g}\) evolves by
\begin{equation}
\label{eq:rf_reparam}
\partial_t \bar{g}_t = - 2\bar{\ric}_t + \lie_{\bar{\xi}_t} \bar{g}_t
\end{equation}
where
\[
\xi_t = (\phi_t)_{\ast} \bar{\xi_t}
\]
so that,
\[
\bar{\xi}_t = (\phi_t^{-1})_{\ast} \xi_t = \frac{1}{2} \grad_{\bar{\ric}_t} \bar{\R}.
\]

For any time dependent tensor, we have the relation,
\begin{equation}
\label{eq:dt_reparam}
\begin{split}
\partial_t \bar{T}_t &= \partial_t \phi_t^{\ast} T_t = \phi_t^{\ast} \partial_t T_t + \phi_t^{\ast} \lie_{\xi_t} T_t \\
&= \overline{\partial_t T_t} + \lie_{\bar{\xi}_t} \bar{T}_t.
\end{split}
\end{equation}

Applying \eqref{eq:dt_reparam} to \(\bar{u}\) and using the evolution of \(\R\) \eqref{eq:R_ev},
\[
\partial_t \bar{u} = \frac{1}{\bar{\R}} \left(\overline{\laplace} \bar{\R} + 2\abs{\bar{\ric}}^2\right) + \lie_{\bar{\xi}} \bar{u}.
\]
But,
\[
\begin{split}
\lie_{\bar{\xi}} \bar{u} &= d\bar{u} \cdot \bar{\xi} = \bar{\ric} \left(\bar{\xi}, \overline{\grad}_{\bar{\ric}} \bar{u}\right) \\
&= -\frac{1}{2} \bar{\ric} \left(\overline{\grad}_{\bar{\ric}} \bar{\R}, \overline{\grad}_{\bar{\ric}} \bar{u}\right) \\
&= -\frac{\bar{\R}}{2} \abs{\overline{\grad} \bar{u}}_{\bar{\ric}}^2
\end{split}
\]
so that
\begin{equation}
\label{eq:dtu_reparm}
\partial_t \bar{u} = \frac{1}{\bar{\R}} \left(\overline{\laplace} \bar{\R} + 2\abs{\bar{\ric}}^2\right) - \frac{\bar{\R}}{2} \abs{\overline{\grad} \bar{u}}_{\bar{\ric}}^2 = \bar{Q}.
\end{equation}

\subsection{Maximum Principle For \(\bar{Q}\)}
\label{subsec:harnack_Q}

We just saw that \(\bar{Q}\) does not contain the additional term in \(Q\) resulting from the evolution of the metric. However, we have achieved this at the expense of introducing additional terms into the evolution of \(g\) and so too \(\Rm, \ric, \R\) etc. In the end, effectively we must compute the evolution of the same quantity,
\[
\frac{1}{\bar{\R}} \left(\overline{\laplace} \bar{\R} + 2\abs{\bar{\ric}}^2\right) - \frac{\bar{\R}}{2} \abs{\overline{\grad} \bar{u}}_{\bar{\ric}}^2.
\]
What we have gained is that the computation is easier in the reparametrised flow. The proof of the Harnack inequality is obtained from the following Proposition.

\begin{prop}
\label{prop:Q_maxprinciple}
\[
(\partial_t - \bar{\Delta}) \bar{Q} \geq - \frac{1}{t} + \bar{g}(\overline{\grad} \bar{Q}, X)
\]
where
\[
X = .
\]
\end{prop}

Let us begin with the easier task of dealing with,
\[
P = \frac{1}{\R} \left(\laplace \R + 2\abs{\ric}^2\right) = \frac{\partial_t \R}{\R}.
\]
Despite our claims above about using the reparametrisation to simplify computations, for this part it is simpler to compute \(\partial_t P\) in the original parametrisation and then convert to the reparametrised flow.

\begin{lemma}
\label{lem:dtP}
\[
\partial_t P = \laplace P + g(\grad P, \cdot) + \cdots
\]
\end{lemma}

\begin{proof}
Recall the equations,
\begin{align*}
\partial_t \R &= \laplace \R + 2 \abs{\ric}^2 \tag{\ref{eq:R_ev} revisited}, \\
[\partial_t, \laplace] f &= 2 g(\ric, \nabla^2 f) \tag{\ref{eq:R_commutator} revisited}.
\end{align*}

Then we compute,
\[
\begin{split}
\partial_t P &= \frac{1}{\R} \left[\partial_t \laplace \R + 2 \partial_t \abs{\ric}^2\right] - \frac{\partial_t \R}{R^2} \left[\laplace \R + 2 \abs{\ric}^2\right] \\
&= \frac{1}{\R} \laplace \partial_t \R + \frac{2}{\R} g(\ric, \nabla^2 \R) + \frac{2}{\R} \partial_t \abs{\ric}^2 - P^2.
\end{split}
\]

On the other hand,
\[
\nabla P = \partial_t \nabla u + [\nabla, \partial_t] u
\]
and
\[
\begin{split}
\laplace P =& \partial_t \laplace \ln \R + [\laplace, \partial_t] \ln \R \\
=& \partial_t \left(\frac{1}{\R} \laplace \R - g(\nabla \ln \R, \nabla \ln \R)\right) + 2g(\ric, \nabla^2 \ln \R) \\
=& \frac{1}{\R} \partial_t \laplace \R - \frac{\laplace \R}{\R} P - 2 g(\partial_t \nabla \ln \R, \nabla \ln \R) + 2\ric(\nabla \ln \R, \nabla \ln \R) + 2g(\ric, \nabla^2 \ln \R) \\
=& \frac{1}{\R} \partial_t \laplace \R - \frac{\laplace \R}{\R} P - 2 g(\nabla P, \nabla \ln \R) \\
& + 2 g([\nabla, \partial_t] u, \nabla \ln \R) + 2\ric(\nabla \ln \R, \nabla \ln \R) + 2g(\ric, \nabla^2 \ln \R) 
\end{split}
\]

Thus,
\[
\begin{split}
(\partial_t - \laplace) P =& 2 g(\nabla \ln \R, \nabla P) + \frac{\laplace \R}{\R} P - P^2 \\
& - 2 g([\nabla, \partial_t] u, \nabla \ln \R) - 2\ric(\nabla \ln \R, \nabla \ln \R) + \frac{2}{\R} \partial_t \abs{\ric}^2 \\
& + \frac{2}{\R} g(\ric, \nabla^2 (\R - \ln \R)).
\end{split}
\]
\end{proof}

\begin{proof}[Proof of \Cref{prop:Q_maxprinciple}]

\end{proof}

\subsection{Proof Of The Harnack Inequality}
\label{subsec:harnack_proof}

Since \(Q \geq -1/t\) if and only if \(\bar{Q} \geq -1/t\), the Harnack inequality in the reparametrised setting now looks more like the original Li-Yau differential Harnack \cite{LiYau:/1986}. Then by the choice of reparametrisation, leading to equation \eqref{eq:dtu_reparm} and by \Cref{prop:Q_maxprinciple}, the Harnack inequality follows directly.

\begin{proof}[Proof of \Cref{thm:harnack}]
\Cref{prop:Q_maxprinciple} above gave us
\[
(\partial_t - \bar{\Delta}) \bar{Q} \geq - \frac{1}{t} + \bar{g}(\overline{\grad} \bar{Q}, X).
\]
Let \(\bar{Q}_0 = \min\{0, \inf_{x\in M} \bar{Q}(x, 0)\}\), and let \(q(t) = -\tfrac{1}{t - \bar{Q}_0^{-1}}\) be the solution of \(\partial_t q = q^2\), \(q(0) = -\bar{Q}_0^{-1}\). Then the maximum principle gives,
\[
\bar{Q} \geq -\frac{1}{t - \bar{Q}_0^{-1}} \geq -\frac{1}{t}
\]
for all \(t > 0\).

Converting back,
\[
Q = \varphi_t^{-1}){\ast} \bar{Q} \geq -\tfrac{1}{t}
\]
proving equation \eqref{eq:harnack_equiv} which is equivalent to the Harnack inequality stated in the theorem.
\end{proof}

\end{document}
